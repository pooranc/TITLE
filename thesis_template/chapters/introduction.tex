\chapter{Introduction}\label{chap:introduction}
Migration is one of the social phenomena in human beings life. It is the physical movement of human beings from one place to another for short or long term within or state borders. The sociologist has come up with the push-pull model after lots of research. This model tells or compares the push factors, that usually happens when sending country forces people to leave their homes, to pull factors, which attracts the people onto receiving country. In recent history, the reasons for migration has changed and its changing. Today people migrate as an asylum seeker or as a trafficked person or refugee or as an international student.

Data Regarding migration flows are largely inconsistent across countries, typically outdated and
are often nonexistent. Research on migration policy making from International Organization for
Migration(IOM)  \cite{IOM's} quotes "Migration has become one of the most challenging issues confronting
policymakers around the world. The growing complexity of internal and cross-border human mobility
has highlighted the need for reliable and timely data to inform migration policy development
and humanitarian assistance". 

Migration data is one the important parameter for demographic changes, But the traditional methods of collecting and storing data of migration statistics are limited. The census which is the comprehensive source of information about the entire population usually takes place every 5 to 10 years, some data is lost between census cycles. Methods like census and population registers may often not be appropriate to estimate short-term migration. 

Spread of internet application, social media, and mobile phones as massively effected on how we communicate with others. \cite{Christina_Hughes_et_al.} discusses about the traditional and new approaches of gathering human migration data. This has helped many researchers to study on societies and due to the Digital revolution, these data are stored. The key features in these data are Time and Geo-location. Availability of Spatial-Temporal data from online sources has opened up new opportunities for the data scientist. These data can be accessed by anyone with relevant technical skills like Twitter data-sets, Linked-In data-sets, mobile phone data or can be bought from companies like Google and Facebook (they don't sell the user data). Mainly, the data collected is based on the data desired. "Time" would be one of the important parameters to estimate the migration trends. Along with that, a specific group of population data (example: "Refugee") could be used to analyze trends. Human migration flow based on political reason, short-term migration for Education, employment and gender can be used to collect data. Twitter is widely accepted all over the world and people use this forum to communicate all types of views and opinion, in this research, the extraction of the topic specifically to human migration from a tweet is studied. 


\cite{Cortis} has proposed a successful approach for identifying tweets associated with the subject "cyberbullying". Our approach differs from the approach mentioned, as the subject of the tweet is human migration. Since Twitter is widely used in all strata, it can be regarded as a good picture of what is happening worldwide.  Our assessment shows that people express their opinions on all trending topics. Of all topics, politics, war and human migration are the focus of this research. These data can be analyzed to understand the opinion of the people who tweet about human migration.  With the focus on mining Twitter data and gathering information on a specific topic which is Human migration in my research. I need to understand why people migrate and implement a strategy to mine migration tweets. Once the Tweets are classified as migration tweets, these tweets are used to understand the sentiment of the people who are tweeting this. So the work in my thesis concentrates on following questions.


  

\begin{itemize}
  \item A framework to classify Tweets which are related to human migration.
  \item Detect the sentiment of the migration classified tweets. 
\end{itemize}

This paper is organized as follows: chapter \ref{chap:background}  provides the background of the different related fields.
The main topics that are discussed are: Twitter data analysis, Text mining and classification algorithms and related work and in chapter \ref{chap:approach} we describe problem formulation and methods used in this research.
approach and its concrete implementation. The results are
presented in Section 4, accompanied by an empirical analysis of computer scientists on Twitter in Section 5. We draw
conclusions about our approach in Section 6.


In this 


\underline{brief agi bari about the approach}

