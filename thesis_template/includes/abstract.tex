% !TeX spellcheck = en_US
% !TeX encoding = UTF-8
\chapter*{Abstract}
The objective of this thesis is to initially classify the tweets as human migration tweet or not. And then detect the sentiment of the classified migration tweet.  People use the tweets, which are a short message to express their opinion on the topic they are interested in. Which makes, interesting and challenging to classify these tweets to specific classes. Migration classifier classifies tweets into "migrations" or "not migrations" classes and sentiment classifier classifies the sentiment of the tweet into "positive" or "negative" classes. The binary classifier is built using a supervised machine learning algorithm.

Set of supervised classifier algorithm consisting of logistic regression, naive Bayes, decision trees are used to create the classification model.  However, the data required to create these classifiers should be preprocessed and features extracted. The count vector and the presence of migration terms are used as the feature vector for the migration detection model, while only the count vector is used as the feature vector for the sentiment detection model. The performance of the learned classifier model is then calculated using the test data.

The dataset for the migration detection model is self-built by searching the data archive using hashtags and manually annotating the labels.  An annotated dataset with the positive and negative labels is used for sentiment analysis. Accuracy and the F1-score of both the models were pretty good, however, when the migration tweets from the migration detection model are sent to sentiment detection model, the prediction was approximately 55\% accurate.

\textit{Keywords: Data analysis, text mining, classification, sentiment analysis, twitter}